\documentclass[12pt,letterpaper,titlepage,twoside]{article}
\usepackage{atkirtland}
\begin{document}
% category: [math]
% categories: [math, physics]
% tags: [math, tensors, physics, differential]
\title{Tensors}
\author{Aaron Kirtland}
\date{\today}
\titlepage{}
\tableofcontents

\section{Tensors and Maxwell's Equations}

\subsection{Introduction}

Here I'll be building up to interpreting Maxwell's equations with tensors.
I'm using the following references:

\begin{itemize}
  \item \url{https://www.ese.wustl.edu/~nehorai/Porat_A_Gentle_Introduction_to_Tensors_2014.pdf}
  \item Wikipedia
\end{itemize}

At times, I may include information not directly pertinent to the goal, but just because it may be helpful to compare definitions to other previously-studied topics.

\subsection{Tensor}

\begin{definition}[tensor]
A $(p,q)$ tensor $T∈ \underbrace{V⊗ … ⊗ V}_{p\text{copies}} ⊗ \underbrace{V^* ⊗ … ⊗ V^*}_{q \text{copies}}$
\end{definition}

\subsection{Operations on tensors}

\begin{definition}[contraction]
Let $T$ have type $(a,b)$.
The $(i,j)$th contraction of $T$ is $T_{t_1,…,t_{j-1},k,t_{j+1},…,t_b}^{s_1,…,s_{i-1},k,s_{i+1},…,s_a}$
\end{definition}

\begin{definition}[raising and lowering]
\end{definition}

\begin{definition}[Levi-Civita symbol]
  \[ε_{i_1,…,i_n}=\begin{cases}1 & i_1,…,i_n \text{even} \\ -1 & \text{odd} \\ 0 & \text{not a permutation}\end{cases}\]
\end{definition}

\begin{example}
  \[\det(A)=ε_{i_1,…,i_n}A^1_{i_1}⋯ A^n_{i_n}\]
\end{example}

\subsection{Beginning}

\begin{definition}[Gram Matrix]
Let $B$ be a bilinear form on a vector space $V$.
The Gram Matrix $G$ is defined by $G_{ij}=B(v_i,v_j)$.
\end{definition}

One property of the gram matrix is that $\{v_i\}$ are linearly independent iff $\det(G)≠0$.
I remember learning about this matrix once in machine learning class as well in the context of kernel functions.

\subsection{Equivalence relations on matrices}

\begin{definition}
Two $n×n$ matrices $A$ and $B$ are similar if there exists a $P$ such that $B=P^{-1}AP$, ie if $A$ and $B$ are in the same conjugacy class of $GL(n,𝔽)$. 
\end{definition}

Similar matrices are the same matrix with respect to a different basis, and this similar matrices share all non-basis-dependent properties.

\begin{definition}[Matrix congruence]
Let $A$ and $B$ be $n×n$ matrices over $𝔽$.
$A$ and $B$ are congruent if there exists an invertible $P$ such that $B=P^{T}AP$.
\end{definition}

Matrix equivalence is defined, in general, on rectangular matrices, and is more restrictive than similarity.

\subsection{Back to the program}

\begin{theorem}[Sylvestor]
Every real symmetric matrix Gram matrix $G$ is congruent to a diagonal matrix $Λ$ with entries $0,±1$ that is unique up to congruence.
\end{theorem}
\begin{proof}
  
\end{proof}

\begin{definition}
An $n×n$ symmetric real matrix $M$ is positive semidefinite if for all $x∈ ℝ ^n$, $x^{T}Mx≥0$.
If $x^{T}Mx=0⇒ x=0$, then $M$ is positive definite.
Negative semidefinite and negative definite matrices are defined similarly.
\end{definition}

Equivalently, a positive definite matrix has all positive eigenvalues, a positive semidefinite matrix has all nonnegative eigenvalues, and so on.

TODO proof

\begin{definition}[signature]
Let $n^+,n^-,n^0$ be the number of $+1,-2,0$, respectively, in $Λ$.
$(n^+,n^-,n^0)$ is called the signature of $G$.
\end{definition}

When $\{V\}$ are linearly independent, then $G$ is positive-definite, so because it is symmetric, 

\begin{theorem}
If $G$ is positive, then it is positive definite.
\end{theorem}
\begin{proof}
\begin{equation*}
\begin{split}
  x^{T}Gx
&=∑_{i,j}x_i G_{ij} x_j \\
&=<∑_i x_i v_i,∑_j x_j v_j> \\
&=\Vert ∑_i v_i x_i \Vert^2 \\
&≥0
\end{split}
\end{equation*}
\end{proof}

Note that $B$ is a bilinear form, not a Hermitian inner product.
It turns out that if $G$ is positive, then the space is Euclidean.

TODO elaborate

\subsection{The metric tensor}

\begin{definition}[metric tensor]
The metric tensor $g_{ij}$ of an inner product space is a $(0,2)$ tensor with coordinates under $\{v\}$ given by the Gram matrix.
\end{definition}

Because $G$ is nonsingular, there is a dual metric tensor $g^{ij}$ that satisfies $g_{ij}g^{jk}=δ_i^k$

Why are tensors multiplies like matrices? TODO


\subsection{types of potentials}

\begin{definition}[scalar potential]
A scalar potential is a vector field whose gradient $v=-∇ Φ$ is a vector field. 
Then for $Φ$ $C^1$ on $U$, $v:U→ ℝ ^n$, where $U⊂ ℝ ^n$ is open, is said to be conservative and curl-free as the curl vanishes everywhere.
\end{definition}

\begin{definition}[Vector potential]
A vector potential is a vector field $A$ whose curl $∇ ×A$ is a vecctor field.
\end{definition}

\begin{theorem}
Let $v:ℝ ^3→ ℝ ^3$ be a twice continuously differentiable solenoidal vector field, and that $v(x)$ decreases sufficiently fast as $||x||→ ∞$.
Then $A(x)=\frac{1}{4π}∫_{ℝ ^3} \frac{∇ _y×v(y)}{\Vert x-y\Vert}d^3y$
is a vector potential for $v$.
\end{theorem}

This contrustuction can be generalized for an arbitrary Helmholtz decomposition.
This is not unique, as for any scalar function $m∈ C^1$, $v=A+∇ m$.

\begin{question}
Do there exist scalar potentials that are not of the above form, $A+∇ m$?
\end{question}

This also relates to ``gauge fixing''.

\begin{theorem}[Helmholtz decomposition]
Let $F$ be a vector field on a bounded $V⊆ ℝ ^3$ that is twice continuosly differentiable.
Then there exists a scalar potential $Φ$ and vector potential $A$ such that $F=-∇ Φ+∇ × A$.
\end{theorem}
\begin{proof}
TODO
\begin{itemize}
  \item what is differentiability over a vector field
  \item \url{https://en.wikipedia.org/wiki/Helmholtz_decomposition}
\end{itemize}
\end{proof}

$-∇ Φ$ is also called the longitudinal partof $F$, and $∇ × A$ is also called the transverge part of $F$.

\subsection{electromagnetic potentials}

\begin{definition}
The magnetic vector potential $A$ is a vector potential of the magnetic field $B$
\end{definition}

\begin{definition}[electric potential]
Let $F=E+\frac{∂A}{∂t}$.
By Faraday's law, $F$ is conservative.
TODO why is this? \url{https://en.wikipedia.org/wiki/Electric_potential}
Therefore, $E=-∇ V-\frac{∂A}{∂t}$, where $V$ is the scalar potential defined by $F$.
\end{definition}

Note that in the electrostatic case, we have that $V$ is a scalar potential of $E$.
Also, we will be using both $V$ and $φ$ to refer to the electric potential unless these symbols are otherwise defined.

\begin{definition}[electromagnetic four-potential]
$A^α=(φ/c,A)$.
\end{definition}

\begin{definition}[four-gradient]
$∂_μ=(\frac{1}{c}\frac{∂}{∂t},∇ )=(\frac{∂_t}{c},∇ )$
\end{definition}

\begin{definition}[category of manifolds]
$\Man^p$ is the category whose objects are manifolds of smoothness class $C^p$ and whose morphisms are $p$-times differentiable maps.
Similarly, the category of smooth manifolds is $\Man^∞$ and the category of analytic manifolds is $\Man^ω$.
\end{definition}

TODO what does it mean looking at manifolds modeled on a fixed caegory $A$?

\subsection{category theory}

Here I'll try to build up to categorically describing a differential form.

\url{https://ncatlab.org/nlab/show/differentiable+manifold}
\url{https://ncatlab.org/nlab/show/tangent+bundle}

\begin{definition}[bundle]
A bundle over an object $B$ in a category $C$ is an object $E$ of $C$ together with a morphism $p:E→ B$.
\end{definition}

\begin{definition}[section]
  \url{https://ncatlab.org/nlab/show/section}
\end{definition}

\begin{definition}[tangency relation]
\end{definition}

\begin{definition}[tangent vector]
A tangent vector on $X$ at $x$ is an equivalence class of the tangency equivalence relation, and the set of all tangent vectors at $x∈ X$ is denoted $T_x X$.
\end{definition}

\begin{definition}
A coordinate chart is a map $φ:ℝ ^n\xrightarrow[≅ ] X|_{\im φ}↪ X$
\end{definition}

\begin{theorem}
$T_x X$ is a real vector space.
\end{theorem}
\begin{proof}
\end{proof}

\begin{definition}[exterior algebra]
  \url{https://ncatlab.org/nlab/show/exterior+algebra}
\end{definition}

\begin{definition}[differential form]
  \url{https://ncatlab.org/nlab/show/differential+form}
\end{definition}

This content is necessary for seeing pullbacks generally, but might not be immediately necessary for our goal.

\begin{definition}[fiber]
The fiber of a morphism of bundle $f:E→ B$ over a point $x$ of $B$ is the collection of generalized elements of $E$ that are mapped by $f$ to $x$.
\end{definition}

\begin{definition}[pullback]
  
\end{definition}

TODO
\begin{itemize}
  \item \url{https://ncatlab.org/nlab/show/pullback}
  \item \url{https://ncatlab.org/nlab/show/fiber}
  \item \url{https://ncatlab.org/nlab/show/bundle}
  \item \url{https://ncatlab.org/nlab/show/differential+form}
\end{itemize}

\subsection{TODO}

\begin{itemize}
  \item \url{https://en.wikipedia.org/wiki/Electromagnetic_tensor}
  \item \url{https://en.wikipedia.org/wiki/Electromagnetic_four-potential}
  \item \url{https://en.wikipedia.org/wiki/Exterior_derivative#Exterior_derivative_of_a_k-form}
\end{itemize}

\end{document}
